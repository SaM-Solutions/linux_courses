\begin{frame}[fragile]
  \frametitle{strace, ltrace}
  \begin{center}
   Основные опции 
  \end{center}
  \begin{itemize}
   \item \texttt{-e <expression>}
\begin{lstlisting}[language=sh]
  strace -e read,write -e read=50 
\end{lstlisting}
   \item \texttt{-p <pid>}
\begin{lstlisting}[language=sh]
  ltrace -p 1021
\end{lstlisting}
   \item \texttt{-c}
   \item \texttt{-f} следить за дочерними процессами
   \item \texttt{-o <filename>}
   \item \texttt{-t, -T}
  \end{itemize}
\end{frame}

\begin{frame}
 \frametitle{Некоторые возможные применения}
 \begin{itemize}
  \item Обнаружение файлов, открываемых приложением (\texttt{-e open})
    \begin{itemize}
      \item библиотек, подключаемых в реальном времени
      \item конфигурационных файлов
    \end{itemize}
   \item Обнаружение сетевых соединений \texttt{connect, accept, recvfrom, sendto, poll, read, write} 
   \item Чем занято приложение в текущий момент
   \item Почему приложение сейчас тормозит
   \item Профилирование
   \item Получение дампов записей в устройства, сокеты и т.п.
 \end{itemize}
\end{frame}

\begin{frame}
 \frametitle{Упражнение}
 \begin{enumerate}
   \item Выяснить, какие конфигурационные файлы пытается открыть \texttt{vim}
   \item В каком системном вызове проводит больше всего времени \texttt{ls /usr/bin/}. То же для \texttt{ls -l /usr/bin}
 \end{enumerate}
\end{frame}
