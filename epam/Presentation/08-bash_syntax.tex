\documentclass[ignorenonframetext, professionalfonts, hyperref={pdftex, unicode}]{beamer}

\usetheme{Copenhagen}
\usecolortheme{wolverine}


\title[bash]{Bourne again shell}
\author[Epam]{{\bf Epam}\\Embedded solutions department}

%Packages to be included

\usepackage[russian]{babel}
\usepackage[utf8]{inputenc}
\usepackage[T1]{fontenc}


\usepackage[orientation=landscape, size=custom, width=16, height=9.75, scale=0.5]{beamerposter}


\usepackage{textcomp}


\usepackage{beamerthemesplit}

\usepackage{ulem}

\usepackage{verbatim}

\usepackage{ucs}


\usepackage{listings}
\lstloadlanguages{bash}

\lstset{escapechar=`,
	extendedchars=false,
	language=C, 
	tabsize=2, 
	columns=fullflexible, 
%	basicstyle=\scriptsize,
	keywordstyle=\color{blue}, 
	commentstyle=\itshape\color{brown},
%	identifierstyle=\ttfamily, 
	stringstyle=\mdseries\color{green}, 
	showstringspaces=false, 
	numbers=left, 
	numberstyle=\tiny, 
	breaklines=true, 
	inputencoding=utf8,
	keepspaces=true,
	morekeywords={u\_short, u\_char, u\_long, in\_addr}
	}

\definecolor{darkgreen}{cmyk}{0.7, 0, 1, 0.5}

\lstdefinelanguage{diff}
{
    morekeywords={+, -},
    sensitive=false,
    morecomment=[l]{//},
    morecomment=[s]{/*}{*/},
    morecomment=[l][\color{darkgreen}]{+},
    morecomment=[l][\color{red}]{-},
    morestring=[b]",
}




\begin{document}





%%%%%%%%%%%%%%%%%%%%%%%%%%%%%%%%%%%%%%%%%%%%%%%%%
%%%%%%%%%% Begin Document  %%%%%%%%%%%%%%%%%%%%%%
%%%%%%%%%%%%%%%%%%%%%%%%%%%%%%%%%%%%%%%%%%%%%%%%%



\begin{frame}
	\frametitle{BASH}
	\titlepage
	\vspace{-0.5cm}
	\begin{center}
	%\frontpagelogo
	\end{center}
\end{frame}

\begin{frame}
	\tableofcontents
%	[hideallsubsections]
\end{frame}



%%%%%%%%%%%%%%%%%%%%%%%%%%%%%%%%%%%%%%%%%   
%%%%%%%%%% Content starts here %%%%%%%%%%
%%%%%%%%%%%%%%%%%%%%%%%%%%%%%%%%%%%%%%%%%

\section{Тесты и сравнения}

\mode<all>{%%

\begin{frame}
\frametitle{Общий синтаксис}

	Условие:
	\begin{itemize}
		\item Exit status любой программы
		\item test или $[$ 
		\item Двойные скобки {\bf (( ... ))} и конструкция {\bf let}
	\end{itemize}


	Конструкция для сравнения:
	\begin{itemize}
		\item \&\& и ||
		\item if/then/else
	\end{itemize}

\end{frame}


\begin{frame}[fragile]
\frametitle{test}

	\begin{itemize}
	    \item ! -- отрицание
	    \item -z СТРОКА
	    \item СТРОКА1 = СТРОКА2
	    \item СТРОКА1 != СТРОКА2
	    \item ЦЕЛОЕ1 -eq ЦЕЛОЕ2
	    \item ЦЕЛОЕ1 -ge ЦЕЛОЕ2
	    \item ЦЕЛОЕ1 -lt ЦЕЛОЕ2
	    \item -d ФАЙЛ
	    \item -e ФАЙЛ
	    \item -f ФАЙЛ
	\end{itemize}

\end{frame}


\begin{frame}[fragile]
\frametitle{Пример}

	Написать скрипт сравнивающий переменную окружения с заранее заданным значением:
	
	\small\begin{lstlisting}
#!/bin/bash

VAR=$1
STRING=test
[ $STRING == $VAR ] && echo "Строки одинаковы" || echo "Строки разные"
exit
	\end{lstlisting}
    \normalsize
	И запустить этот скрипт:
	
	\begin{enumerate}
		\item {\tt sh script.sh string}
		\item {\tt sh script.sh test}
		\item {\tt sh script.sh}
	\end{enumerate}

\end{frame}

\begin{frame}
	\frametitle{Пример: варианты исправления}

		\begin{enumerate}
			\item {\tt [ ``\$STRING'' == ``\$VAR'' ] }
			\item {\tt [ z\$STRING == z\$VAR ] }
		\end{enumerate}

\end{frame}

\begin{frame}[fragile]
\frametitle{\&\& и ||}
	Синтаксис:
\begin{verbatim}
условие && true || false
\end{verbatim}

	\pause
	Пример:
\begin{lstlisting}[language=bash]
test -z "$DISPLAY" && echo "text mode" || echo "graphical mode"
\end{lstlisting}
	
	\pause

	\begin{itemize}
	    \item Запустить команды "true" и "false"
	    \item В случае успеха вывести "успех"
	    \item В случае неуспеха вывести "неудача"
	\end{itemize}
\end{frame}

%\begin{frame}[fragile]{}

%\end{frame}

}
%\section{Операторы}

%\mode<all>{%%


\begin{frame}{}

\end{frame}
}

\section{Арифметические операции}

\mode<all>{%% Arithmetic

\begin{frame}
  \frametitle{}
  \begin{itemize}
   \item  Конструкция {\tt ((...))}
    \begin{block}{Примеры}
     {\tt (( a=10 )); echo \$(( a++ )); echo \$a; } 
    \end{block}
   \item  {\tt let}
    \begin{block}{Примеры}
     {\tt let a=10; echo \$a; let a+=-2; echo \$a; echo \$(( $--$a)); echo \$a} 
    \end{block}
   \item  {\tt expr } внешняя команда 
    \begin{block}{Пример}
       { \tt x=\`{}expr \$x + 1\`{} }
    \end{block}
  \end{itemize}
\end{frame}

\begin{frame}[fragile]
\frametitle{Операторы в арифметических выражениях}
\begin{enumerate}
\item Инкременты, декременты {\tt id++, id$--$, ++id, $--$id } 
\item Арифметические операторы {\tt **,*,/,\%,+,-} 
\item Побитовые операторы {\verb+ ~,>>,<<,^,&,|+}
\item Операторы сравнения {\tt <=,>=,<,>, ==, !=}
\item Логические операторы {\tt \&\&, || } 
\item Тернарный оператор {\tt expr ? expr : expr }
\item Операторы присваивания
\begin{lstlisting}[language=C]
=, *=, /=, %=,
+=, -=, <<=, >>=,
&=, ^=, |=  
\end{lstlisting}
\end{enumerate} 

  
\end{frame}


\begin{frame}[fragile]
  \frametitle{Упражнение}
 \begin{enumerate}
   \item {\bf Конец света:} Вывести дату конца юниксовых времен ( время $2^{31}-1$) {\tt date -d @<seconds> } 
   \item Проверить результат арифметической операции (( 5>10 ))
 \end{enumerate}
\end{frame}
}

\section{Циклы}

\mode<all>{\begin{frame}
\frametitle{Основные конструкции для циклов}
  \begin{itemize}
   \item while
   \item for
   \item until
   \item break, continue 
  \end{itemize}
\end{frame}

\begin{frame}[fragile]
  \frametitle{Циклы for}
  \begin{enumerate}
    \item Стандартная форма
\begin{lstlisting}[language=sh,frame=single]
  for x in list 
  do
    op1
    op2
  done
\end{lstlisting}
    \item Арифметическая форма
\begin{lstlisting}[language=sh,frame=single]
  for ((i=0;i<10;i++)) 
  do
    op1
    op2
  done
\end{lstlisting}
  \end{enumerate}
\end{frame}
\begin{frame}[fragile]
\frametitle{}
  \begin{block}{Примеры}
\begin{lstlisting}[language=sh,frame=single]
for file in *
 do md5sum $file
done

for ((i=0;i<10;i++))
 do echo $i
done
\end{lstlisting}
  \end{block}
\end{frame}

\begin{frame}[fragile]
\frametitle{Циклы while,until}
\begin{lstlisting}[language=sh,frame=single]
while expr1; ... exprN
do
 op
done
\end{lstlisting}
\end{frame}

\begin{frame}[fragile]
\frametitle{}
\begin{block}{Пример}
\begin{lstlisting}[language=sh,frame=single]
while ((i++))
     read y
do
 echo $i $y
done
\end{lstlisting}
\end{block}
\begin{block}{Пример}
\begin{lstlisting}[language=sh,frame=single]
while :
do
 x=$RANDOM
 echo $x
 [[ $x -gt 1100 ]] && break
done
\end{lstlisting}
\end{block}
\end{frame}

\begin{frame}[fragile]
\frametitle{Упражнения}
\begin{enumerate}
\item Посчитать сумму кубов чисел от 1 до 100
\item Вывести в файл 10 случайных чисел от 0 до 80
\item Построить гистограмму данных из предыдущего файла файла {\bf Hint:} {\tt while read, echo -n }
\end{enumerate}
\end{frame}
}

\section{Условные операторы}

\mode<all>{
\begin{frame}[fragile]
\frametitle{Синтаксис {\bf if}}

	\begin{columns}
		\column{0.4\textwidth}
	
	\begin{lstlisting}[language=bash]
if [ условие1 ]
then
   . . .
elif [ условие2 ]
then
   . . .
else
   . . .
fi
\end{lstlisting}
		\column{0.6\textwidth}
	{\bf Практическое задание:} \\
	\begin{itemize}

		\item с помощью конструкции {\bf if} проверить существует ли файловый объект передаваемый в качестве параметра скрипту
		\item если нет, то создать директорию с таким именем
		\item если cуществует и файл является shell-скриптом, то запустить его
		\item если существует и является директорией, то вывести на экран первых 5 файлов в этой директории
	\end{itemize}
	\end{columns}
\end{frame}

\begin{frame}[fragile]
\frametitle{Условные операторы: case}
\begin{itemize}
\item {\tt case}
\begin{lstlisting}[language=sh,frame=single]
case "$variable" in 
 pattern1) command1
           command2
          ;;
 pattern2|pattern3)
         command3
         command4
        ;;
esac
\end{lstlisting}
\end{itemize}
\end{frame}

\begin{frame}[fragile]
\frametitle{Использование case вместе с getopts}
\begin{lstlisting}[language=sh,frame=single]
while getopts "af:h" Option
do
  case $Option in 
    a) OPTA=1 ;;
    f) OPTFILE=1
       FILENAME=$OPTARG
       ;;
    h) echo "Usage: $0 [-ah] -f <filename>";;
  esac  
done
shift $((OPTIND-1))
\end{lstlisting}
\end{frame}

\begin{frame}[fragile]
\frametitle{Упражнение}
\begin{enumerate}
\item Написать программу, которая по опции {\tt -h } выводит помощь, без опций выводит время в stdout,
с опцией -f выводит время в указаный файл
\end{enumerate}
\end{frame}
}


%\section{Массивы}

%\mode<all>{\input{../../slides/bash/arrays}}

%\section{Функции}

%\mode<all>{\input{../../slides/bash/functions}}

\end{document}
